An automaton can be described by a tuple $A=(Z,S,N,f)$ where $Z$ is a $d\times d$ lattice composed of discrete cells that can hold states $S=\{1,2,3\}$. The transition from one cell value to another 
is given by the transition function $f:S\rightarrow S$. The transition of states for a single cell takes other cells in a neighborhood into consideration. Here we used
the von Neumann neighborhood, described by the set $N(\mathbf{n}_0)=\{\mathbf{n}_k:\sum_i^d|(\mathbf{n}_k)_i-(\mathbf{n}_0)_i\leq 1\}$. The configuration of the automaton
is described by a function $c:Z\rightarrow S$ that assigns a state to each cell. The next state of a cell is given by $c(\mathbf{n}_0)^{t+1}=f(\{\mathbf{n}_m|\mathbf{n}_m\in N(\mathbf{n}_0)\}$.
The next state for a cell is determined through a random selection from the values of its neighboring cells, with the selection probabilities being proportional to the frequencies of those values. 
Therefore, each agent tends to align its state to the state of the majority of its neighbors. To avoid boundary effect, we used periodic boundary conditions.

In our simulations, the grid is randomly initiated with $S_0=\{1,2\}$. The automaton runs for $100$ steps to reach thermalization and the a new state is introduced to $n$ elements of the grid to create $S=S_0\cup\{3\}$.

Introducing the new values to the network can have different effects if the $n$ sites are dispersed or localized. 