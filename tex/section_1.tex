Innovation dynamics are all the processes that induce change to a certain environment. The context of such environment can range from biological, such as viral mutations to scientific innovation resultant from research. Due to its abstractness, innovation can be sometimes difficult to characterize, however it can be broken down into certain main processes, such as emergence, spread and development of information throughout a system of many agents. 

\textcolor{red}{Talk about the first researchers of innovation}
\textcolor{red}{Talk about why different systems, sometimes behave in similar ways, class of universality, Markovian behavior generates such similarities?}

Statistical mechanics is a field of study that is able to break down the spread of information due to the action of many agents into simple yet very useful mathematical models and equations. In this context, we use statistical mechanics and stochastic calculus to elaborate a model that describes many innovation scenarios and allows us to extract behaviors and patterns of a system that innovates. With this we should be able to detect fraudulent data report in certain industries or the emergence of an external effect in a system, for example the effects in economical growth due to the spread of Covid-19.

One of the first researchers to study and model innovation was Frank Bass. In 1969, he proposed a simple but effective model that described the diffusion of ideas, products and technologies throughout a system or population. The model consists of a differential equation that describes the general behavior of adopters. Each adopter may be separated into innovators and imitators and each one of them will have its own innovation degree and adoption speed. Such model proved simple but useful in predicting market adoption dynamics and obsolescence.

The model proposed by Bass was built under probabilistic assumptions regarding the behaviors of individuals, something that is difficult to nail down or to avoid an over fitting of a system. In our mathematical description, we make simpler assumptions: 1. The random behavior of the system as a whole only depends on its current state, creating a Markov chain. 2. The Markov chain can be described as a diffusion process. 3. The adoption increases the number of agents adhering to a certain idea (or technology, or mutation, ...), so it must be proportional to the time. 4. The obsolescence coefficient decreases the number of individuals adhering to an idea, so it must be inversely proportional to the time. With only four simple assumptions followed by many physical systems, we are able to create a very broad model, which is most certainly not an over fitting due to its simplicity. In other words, this set of considerations is the simplest set of considerations one can make regarding a system that simultaneously explain adoption curves over time.

We also use Recurrent Neural Networks (RNNs) another statistical/stochastic tool that is very versatile in analysing time series through a feedback loop. It consists in feeding processed data back to the neural layer in order to change it in a way that it is able to predict future behaviors of a time-series. We use the RNN to better analyse the data gathered from real scenarios and compare it with what the analytical solution for the innovation model describes.

In the following chapter we introduce the model and its analytical solutions, which lead to the average number of agents adhering to a certain idea or information. This chapter is followed by a comparison via fitting between the mathematical model and data gathered from real scenarios, such as viral mutations exhibited by the Covid-19 virus, scientific articles year-over-year citations and the monthly searches of social media in the platform Google-Trends. Afterwards, we enhance the analysis by comparing the model and the data with RNNs and finally, we discuss applications of our mathematical model in predicting growth and abandonment of ideas, platforms, industries, etc. Which leads up to the conclusions and future perspectives in the area.
