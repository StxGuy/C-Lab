The state corresponding to a population of $n$ agents adhering to the new technology is described by the stochastic process $X_n$. The attachment rate per single cell for the new technology is given by $\lambda_n(t)$, whereas the detachment rate is given by $\mu_n(t)$. Moreover, the chance an individual attaches to or detaches from a technology must depend on the number of people already attached to it due to social conformity \cite{herd,EuEcon}. Therefore, we choose $\lambda_n(t)=n\lambda_0(t)$ and $\mu(t)=n\mu_0(t).$

We assume that the process $X_n$ is independent on its history, and is, consequently, Markovian. Therefore $P(X_{n+1}=x_{n+1}|X_1=x_1,X_2=x_2,\hdots,X_n=x_n)=P(X_{n+1}=x_{n+1}|X_n=x_n)$. The process $X_n$ induces a Markov chain described by:

\begin{eqnarray}
    X_n(t+\Delta_t)=&&(1-\lambda_n\Delta-\mu_n\Delta_t)X_n(t)+\lambda_{n-1}(t)\Delta_t X_{n-1}(t)\nonumber\\
    &&+\mu_{n+1}(t)\Delta_t X_{n+1}(t),
\end{eqnarray}
where $\Delta_t$ is a short period. At the limit where $\Delta_t\rightarrow 0$, the previous equation becomes:

\begin{eqnarray}
    \frac{dX_n(t)}{dt}=&&-(\lambda_n(t)+\mu_n(t))X_n(t)+\mu_{n+1}(t)X_{n+1}(t)\nonumber\\
    &&+\lambda_{n-1}(t)X_{n-1}(t).
\end{eqnarray}
The expected number of people attached to the technology at some instant $t$ and its time variation are given by:

\begin{eqnarray}
        M(t)=&&\sum_{n=1}^\infty nX_n(t)\nonumber\\
        \frac{dM(t)}{dt}&&=-\left(\lambda_0(t)+\mu_0(t)\right)\sum_{n=1}^\infty n^2X_n(t)\nonumber\\
        &&+\mu_0(t)\sum_{n=1}^\infty n(n+1)X_{n+1}(t)+\nonumber\\
        &&+\lambda_0(t)\sum_{n=1}^\infty n(n-1)X_{n-1}(t)\nonumber\\
        &&=\left(\lambda_0(t)-\mu_0(t)\right)M(t).\\
\end{eqnarray}

To account for the temporal dynamics of opinions, we propose that the attachment rate may reduce over time, and the detachment rate follows the opposite trend. This can be due to a number of factors, including the development of other technologies, temporal preferences, and the saturation of the novelty factor. \textcolor{red}{[insert references]}. Therefore, we write:

\begin{equation}
    \begin{aligned}
        \lambda_0(t)&=\frac{\alpha}{t}\\
        \mu_0(t)&=\frac{t}{\sigma^2},\\
    \end{aligned}
\end{equation}
where $\alpha$ is a growth exponent, and $\sigma$ is a characteristic time scale related to detachment. 

The time variation of the expected number of people attached to the technology becomes:

\begin{equation}
    \frac{dM(t)}{dt} = \left(\frac{\alpha}{t}-\frac{t}{\sigma^2}\right)M(t).
\end{equation}
The solution of this equation is:

\begin{equation}
    M(t)=M_0t^\alpha e^{-t^2/2\sigma^2}.
    \label{eq:number_of_attachments_over_time}
\end{equation}

Parameters $M_0$, $\alpha$ and $\sigma$ can be found using a standard Levenberg-Marquardt fitting for any specific technology. These parameters allows us to compare how different technologies grow and decay with time.

In the following section, we compare the Markovian Innovation model with existing data to obtain each of the previously mentioned parameters in real life scenarios.

